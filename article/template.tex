\documentclass[9pt]{pnas-new}
% Use the lineno option to display guide line numbers if required.
% Note that the use of elements such as single-column equations
% may affect the guide line number alignment. 

\RequirePackage[english,slovene]{babel} % when writing in slovene
%\RequirePackage[slovene,english]{babel} % when writing in english

\templatetype{pnasresearcharticle} % Choose template 
% {pnasresearcharticle} = Template for a two-column research article
% {pnasmathematics} = Template for a one-column mathematics article
% {pnasinvited} = Template for a PNAS invited submission

\selectlanguage{slovene}
\etal{in sod.} % comment out when writing in english
\renewcommand{\Authands}{ in } % comment out when writing in english
\renewcommand{\Authand}{ in } % comment out when writing in english

\newcommand{\set}[1]{\ensuremath{\mathbf{#1}}}
\renewcommand{\vec}[1]{\ensuremath{\mathbf{#1}}}
\newcommand{\uvec}[1]{\ensuremath{\hat{\vec{#1}}}}
\newcommand{\const}[1]{{\ensuremath{\kappa_\mathrm{#1}}}} 

\newcommand{\num}[1]{#1}

\graphicspath{{./fig/}}

\title{Modeliranje evakuacije ljudi pred napadalcem z uporabo mehke logike}

% Use letters for affiliations, numbers to show equal authorship (if applicable) and to indicate the corresponding author
\author{Martin Božič}
\author{Marija Marolt}
\author{Jakob Maležič}

\affil{Poročilo seminarske naloge pri predmetu Skupinsko vedenje} 

% Please give the surname of the lead author for the running footer
\leadauthor{Novak} 

\selectlanguage{english}

% Please add here a significance statement to explain the relevance of your work
\significancestatement{Crowd evacuation with Assailants via a Fuzzy logic approach}{TODO: Add english abstract?}{Collective behaviour | Crowd evacuation | Fuzzy logic}

\selectlanguage{slovene}

% Please include corresponding author, author contribution and author declaration information
%\authorcontributions{Please provide details of author contributions here.}
%\authordeclaration{Please declare any conflict of interest here.}
%\equalauthors{\textsuperscript{1}A.O.(Author One) and A.T. (Author Two) contributed equally to this work (remove if not applicable).}
%\correspondingauthor{\textsuperscript{2}To whom correspondence should be addressed. E-mail: author.two\@email.com}

% Keywords are not mandatory, but authors are strongly encouraged to provide them. If provided, please include two to five keywords, separated by the pipe symbol, e.g:
\keywords{Skupinsko vedenje | Evakuacija ljudi | Mehka logika } 

\begin{abstract}
%To sem na hitro nekaj nakracal
Vsako leto se po celem svetu zgodi veliko napadov, kjer je ustreljenih in poškodovanih veliko ljudi. V tem članku, s pomočjo mehke logike, zgradimo različne modele, s katerimi lahko simuliramo takšne napade. Najprej predstavimo manjše modele, ki želijo doseči specifičen cilj - izogibanje oviram, iskanje poti in doseganje točke v prostoru. Te modele nato združimo v celoto tako, da jih ustrezno utežimo. Model testiramo in simuliramo v različnih prostorih z različnimi parametri. Za boljše razumevanje simulacije pripravimo tudi uporabniški vmesnik, ki simuliran napad vizualno prikaže. Naš model primerjamo še z ostalimi obstoječimi modeli.
\end{abstract}

\dates{\textbf{\today}}
\program{BM-RI}
\vol{2020/21}
\no{CB:GB} % group ID
%\fraca{FRIteza/201516.130}

\begin{document}

% Optional adjustment to line up main text (after abstract) of first page with line numbers, when using both lineno and twocolumn options.
% You should only change this length when you've finalised the article contents.
\verticaladjustment{-2pt}

\maketitle
\thispagestyle{firststyle}
\ifthenelse{\boolean{shortarticle}}{\ifthenelse{\boolean{singlecolumn}}{\abscontentformatted}{\abscontent}}{}

% If your first paragraph (i.e. with the \dropcap) contains a list environment (quote, quotation, theorem, definition, enumerate, itemize...), the line after the list may have some extra indentation. If this is the case, add \parshape=0 to the end of the list environment.
\dropcap{Z}biranje velikih gruč ljudi na javnih mestih je postalo nekaj neizogibnega. Na avtobusnih postajah, na železniški postaji, v velikih trgovskih centrih ali pa na koncertih in tekmah. Slednje lahko predstavlja veliko nevarnost za ljudi, kot tudi velik izziv za organizatorje in nadzornike takšnih javnih prostorov.

\section*{Metode}
Za simulacijo gruč ljudi smo morali najprej pripraviti model za vodenje posameznega človeka kot tudi napadalca.

V študijah (članek) so pokazali, da je vid glavni vir informacij, na podlagi katerih, se človek v kriznih situacijah odloča o svojih dejanjih. Zato smo tudi mi, zgradili takšne modele, ki se odločajo na podlagi okolice, ki jo posamezen človek vidi.

\subsection*{Mehka logika}
TODO: Dodaj metoda 1.

\subsection*{Umikanje oviram}
TODO: Dodaj metoda 2.

\subsection*{Iskanje poti}
Pri metodi iskanja poti, posamezne osebe, na spremembo trenutne hitrosti in smeri najbolj vplivajo hitrosti in smeri ostalih oseb, v vidnem spektru osebe. Na metodo iskanja poti, bi tako najbolj vplivala oseba, ki se proti izbrani osebi premika z veliko hitrostjo, v nasprotni smeri. Metoda vedno vodi posameznika po najvarnejši poti, skratka se vedno izogiba področjem z visoko negativno energijo. Moč negativne energije se izračunava sproti in je odvisna od trenutnega vpliva ovir in trenutne nevarnosti trčenja posameznika z ostalimi osebki. Kot vpliv ovire v večini upoštevamo zidove, pohištvo in ostale stacionarne reči v vidnem sektorju posameznika. Vpliv ovir označimo z oznako ${(OI^*)}$, pri čemer sistem mehke logike za izračun trenutnega vpliva ovir opišemo z naslednjo enačbo \ref{OI_equation}.

\begin{equation}
\label{OI_equation}
OI^* = R_{1}(\phi^*_{oi}, d^*_{oi})
\end{equation}

Pri čemer ${\phi^*_{oi}}$ predstavlja, v katerem kotu vidnega spektra posameznika se ovira nahaja, ${d^*_{oi}}$ predstavlja razdaljo od posameznika do posamezne ovire. 


\subsection*{Doseganja cilja}
TODO: Dodaj metoda 3.

\subsection*{Utežena vsota}
TODO: Dodaj metoda 3.

\subsection*{Uporabniški vmesnik}
Uporabniški vmesnik bo narejen z uporabo knjižnice p5.js, ki temelji na jeziku Processing in bo prikazoval potek simulacije. Najprej bomo naredili nekaj simulacij v prostorih, ki bodo vnaprej pripravljeni. Nato bomo dodali možnost, da uporabnik sam nariše prostor, v katerem se simulacija izvede.

\section*{Rezultati}
Rezultati bodo predstavljeni s tabelami in grafi, kjer bomo primerjali različne modele evakuacije glede na število žrtev in čas evakuacije. 

\section*{Sklep}
% TODO: Dodaj sklep.

\acknow{JN je napisal uvod in skrbel za jezik, MK je izvedel analizo podatkov in izdelal slike, IC je implementiral algoritem in pognal eksperimente.}
\showacknow % Display the acknowledgments section

% \pnasbreak splits and balances the columns before the references.
% If you see unexpected formatting errors, try commenting out this line
% as it can run into problems with floats and footnotes on the final page.
%\pnasbreak

\begin{multicols}{2}
\section*{\bibname}
% Bibliography
\bibliography{./bib/bibliography}
\end{multicols}

\end{document}